\documentclass[compress, aspectratio=54]{beamer}
%\documentclass[notes=show]{beamer}
%\documentclass[xcolor=dvipsnames]{beamer}
\usepackage[export]{adjustbox}
\usepackage{sidecap}
\usepackage{subfig}
\usepackage{amssymb}
\usepackage{latexsym}
\usepackage{amsfonts}
\usepackage{amsmath}
\usepackage[absolute,overlay]{textpos}
\usepackage[english]{babel}
\usepackage[latin1]{inputenc}
\usepackage{subfig}
%\usepackage{times}
\usepackage[T1]{fontenc}
\usepackage{tabularx}
\newcolumntype{Y}{>{\small\raggedright\arraybackslash}X}
\usepackage{graphicx}
\usepackage{bigstrut}
\usepackage{bbm}
\usepackage{mathrsfs}
\usepackage{epsfig}
\usepackage{array}
%\usepackage{natbib}
\usepackage{hyperref}
\hypersetup{
    colorlinks=true,
    linkcolor=blue,
    filecolor=magenta,      
    urlcolor=cyan,
}
\usepackage{caption}
\usepackage{xcolor}

\mode<presentation> {
%\usetheme[left,width=1.7cm]{Berkeley}
%\usetheme{default}
\usetheme{Boadilla}
  \usecolortheme[RGB={103,102,204}]{structure}
%\usecolortheme{dove}
  \useoutertheme{infolines}
  \setbeamercovered{transparent}
 }

%\usepackage[utf8]{inputenc}

% Default fixed font does not support bold face
\DeclareFixedFont{\ttb}{T1}{txtt}{bx}{n}{12} % for bold
\DeclareFixedFont{\ttm}{T1}{txtt}{m}{n}{12}  % for normal

% Custom colors
\usepackage{color}
\definecolor{deepblue}{rgb}{0,0,0.5}
\definecolor{deepred}{rgb}{0.6,0,0}
\definecolor{deepgreen}{rgb}{0,0.5,0}

\usepackage{listings}

% Python style for highlighting
\newcommand\pythonstyle{\lstset{
language=Python,
basicstyle=\ttm,
otherkeywords={self},             % Add keywords here
keywordstyle=\ttb\color{deepblue},
emph={MyClass,__init__},          % Custom highlighting
emphstyle=\ttb\color{deepred},    % Custom highlighting style
stringstyle=\color{deepgreen},
frame=tb,                         % Any extra options here
showstringspaces=false            % 
}}


% Python environment
\lstnewenvironment{python}[1][]
{
\pythonstyle
\lstset{#1}
}
{}

% Python for external files
\newcommand\pythonexternal[2][]{{
\pythonstyle
\lstinputlisting[#1]{#2}}}

% Python for inline
\newcommand\pythoninline[1]{{\pythonstyle\lstinline!#1!}}
%\renewcommand{\familydefault}{cmss}
%\renewcommand{\mathrm}{\mathsf}
%\renewcommand{\textrm}{\textsf}
\usefonttheme{serif}
\newcommand{\X}{{\mathbf{X}}}
\newcommand{\x}{{\mathbf{x}}}
\newcommand{\E}{\mathsf{E}}
\newcommand{\V}{\mathsf{Var}}

\DeclareGraphicsExtensions{.jpg,.pdf,.mps,.png}

\setbeamercolor{bibliography entry title}{fg=black}
\setbeamercolor{bibliography entry author}{fg=black}
\setbeamercolor{subsection in toc}{fg=structure}
\setbeamercolor{palette primary}{bg=structure, fg=white}
%\setbeamercolor{palette secondary}{bg=structure, fg=black}
%\setbeamercolor{palette tertiary}{bg=structure, fg=black}
\setbeamercolor{caption name}{fg=black} \setbeamersize{text margin
left=.8cm} \setbeamersize{text margin right=1cm}
\hypersetup{linkbordercolor={1 0 0}} \setbeamertemplate{navigation
symbols}{} \setbeamertemplate{headline}[default]

\setbeamertemplate{enumerate items}[default]

\newcounter{transfct}
\newcounter{begbs}
\newcounter{endbs}


\title[Sentiment Analysis]{Machine Learning For Natural Language Processing}

\author[Arieda Mu\c co]{Arieda Mu\c co\footnote{Based on: https://medium.com/analytics-vidhya/simplifying-social-media-sentiment-analysis-using-vader-in-python-f9e6ec6fc52f}}
\institute[CEU]{Central European University}
\date{}

\AtBeginSection[] {
  \begin{frame}<handout:0>
    \frametitle{TOC}
    \tableofcontents[currentsection]
  \end{frame}
}


\AtBeginSection[] {
  \begin{frame}<handout:0>
    \frametitle{TOC}
    \tableofcontents[currentsection]
  \end{frame}
}


\pgfdeclareimage[height=.7cm]{logo}{rgs2}
\logo{\pgfuseimage{logo}}
\begin{document}
\captionsetup[subfigure]{labelformat=empty}

\frame{\titlepage}

%%%%%%%%%%%%%%%%%%%%%%%%%%%%%%%%%%%%%%%%%%%

%----------------------------------------------------------------------------%

\begin{frame}
\frametitle{Sentiment Analysis}

\begin{quote}
If you want to understand people, especially your customers...then you have to be able to possess a strong capability to analyze text.
\end{quote}
\begin{flushright}
--Paul Hoffman, CTO Space-Time Insight
\end{flushright}

\end{frame}
%----------------------------------------------------------------------------%



\begin{frame}

\frametitle{What is Sentiment Analysis?}
\begin{itemize}
\item Sentiment Analysis, or Opinion Mining, is a sub-field of Natural Language Processing (NLP) that identifies and extracts opinions and sentiments from given text
%\item The aim of sentiment analysis is to gauge the attitude, sentiments, evaluations, attitudes and emotions of a speaker/writer based on the computational treatment of subjectivity in a text

\end{itemize}
\end{frame}
%----------------------------------------------------------------------------%

\begin{frame}

\frametitle{Why is sentiment analysis so important?}
\begin{itemize}
\item Majority of this data we have today is unstructured text
\begin{itemize}
\item Sources like emails, chats, social media, surveys, images, videos, articles, and documents. 
\end{itemize}
\item It's challenging
\begin{itemize}

 \item Huge amount of data involved (difficult and time consuming)
 \item The kind of language used in them to express sentiments, i.e., short forms, memes and emoticons
 \end{itemize}
\end{itemize}
\end{frame}

%----------------------------------------------------------------------------%
\begin{frame}
\begin{figure}

\includegraphics[width=0.85\linewidth ]{Figures/emoji.png}
\end{figure}

\end{frame}
%----------------------------------------------------------------------------%


\begin{frame}

\frametitle{Why is sentiment analysis so important?}
\begin{itemize}
\item Sentiment Analysis allows us to make sense out of text by being able to automate this entire process
\end{itemize}
\end{frame}



%----------------------------------------------------------------------------%
\begin{frame}

\frametitle{Why is Sentiment Analysis a Hard to perform Task?}
\begin{itemize}
\item Understanding emotions through text are not always easy. Sometimes even humans can get misled, so expecting a 100\% accuracy from a computer (today) is like asking for the Moon!
\begin{itemize}
\item A text may contain multiple sentiments all at once. For instance: \textit{The intent behind the final season of Game of Thrones was great, but it could have been way better.}
\item Two polarities, i.e., Positive as well as Negative. How do we conclude whether the review was Positive or Negative?
\end{itemize}
\item Computers can't  comprehend Figurative Speech (yet). Can't understand use of similes, metaphors, hyperboles etc qualify for a figurative speech.
\begin{itemize}

 \item \textit{The best I can say about the season finale is that it was interesting.}
 \item The word `interesting' does not necessarily convey positive sentiment and can be confusing for algorithms
 \end{itemize}
\end{itemize}
\end{frame}


%----------------------------------------------------------------------------%


\begin{frame}
\frametitle{Python Libraries}
We are mainly going to use the following libraries 
\begin{itemize}
\item pip install nltk or conda install nltk (if you haven't installed it yet)
\item VADER (Sentiment Analysis)
\begin{itemize}

\item Valence Aware Dictionary and sEntiment Reasoner (VADER) is a lexicon and rule-based sentiment analysis tool that is specifically attuned to sentiments expressed in social media
\item VADER uses a combination of A sentiment lexicon is a list of lexical features (e.g., words) which are generally labelled according to their semantic orientation as either positive or negative
\item For more info: \url{https://www.nltk.org/_modules/nltk/sentiment/vader.html}
\end{itemize}


\end{itemize}
\end{frame}
%----------------------------------------------------------------------------%


%----------------------------------------------------------------------------%

\begin{frame}
\frametitle{VADER}
\begin{itemize}
\item VADER has been found to be quite successful when dealing with social media texts, NY Times editorials, movie reviews, and product reviews. 
\item VADER not only tells about the Positivity and Negativity score but also tells us about how positive or negative a sentiment is
\item The developers of VADER have used Amazon's Mechanical Turk to get most of their ratings, You can find complete details on their \href{https://github.com/cjhutto/vaderSentiment}{GitHub}
\item Link to paper \url{http://comp.social.gatech.edu/papers/icwsm14.vader.hutto.pdf}
\end{itemize}
\end{frame}
%----------------------------------------------------------------------------%

%----------------------------------------------------------------------------%

\begin{frame}
\frametitle{Advantages of using Vader}
\begin{itemize}
\item It works exceedingly well on social media type text, yet readily generalizes to multiple domains

\item It doesn't require any training data but is constructed from a generalizable, valence-based, human-curated gold standard sentiment lexicon
\item It is fast enough to be used online with streaming data

\item  It does not severely suffer from a speed-performance tradeoff
\end{itemize}
\end{frame}
%----------------------------------------------------------------------------%



\begin{frame}
\begin{center}
\huge{Let's try it out!!!}
\end{center}
\end{frame}

\end{document}





